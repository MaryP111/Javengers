\documentclass[12pt]{article}
\usepackage[english,greek]{babel}
\usepackage[utf8x]{inputenc}
\usepackage{graphicx}
\usepackage{fullpage}
\begin{document}

\title{Προδιαγραφές Λογισμικού}
\date{\today}
\author{\selectlanguage{english}Javengers}

\maketitle

\tableofcontents

\section{Εισαγωγή}
\subsection{Ταυτότητα-Επιχειρησιακοί στόχοι}

Το ζητούμενο της εργασίας είναι να αναπτυχθεί ένα διαδικτυακό παρατηρητήριο τιμών, το οποίο θα
επιτρέπει στους χρήστες να καταγράφουν ηλεκτρονικά τις τιμές προϊόντων σε διάφορα καταστήματα. Το
παρατηρητήριο θα λειτουργεί με τη μέθοδο του πληθοπορισμού (crowdsourcing), όπου εθελοντές
επιλέγουν τα προϊόντα και καταγράφουν τις τιμές τους στα καταστήματα ώστε να τις μοιραστούν με
άλλους μέσω μιας δικτυακής υπηρεσίας. 

\subsection{Περίγραμμα επιχειρησιακών λειτουγιών}

$x_{1,2} + x^{x^2} = \frac{y}{y+1}$



\subsection{Αναφορές - Πηγές πληροφοριών}



\end{document}

